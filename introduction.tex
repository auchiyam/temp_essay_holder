\section{Introduction}

This summer, from May 31st to July 28th, I was in Japan to participate in an internship program hosted by Aoyama Gakuin University.  There, I learned various different skills that I hope to use in the future as a computer scientist.

\subsection{Details}

This internship program, hosted by Aoyama Gakuin University in Tokyo, Japan, is a research based program that is aimed towards helping the students improve their skills.  The program I participated in stated that I must learn and contribute to the programming language Ruby or Ruby on Rails in 2 months.  The professor of the lab I was assigned to is Professor Martin D\"{u}rst, who is a computer scientist that specializes in optimizing Unicode in programming languages.  He is also a committer of Ruby and frequently participates in conferences based on his skills, like Ruby Kaigi.  

Since the job description was vague, I was told to come up with some project idea and implement them during the internship program.  There, I worked on a project called domain, which is a type checking system based on the mathematic concept of domain.  I chose this project because I am interested in the development of programming languages, and because I was provided with a chance to contribute to the programming language itself, I decided that it was a good time to challenge myself and implement something I always wanted to see in programming languages.

The timeline of this internship looked like the following:

\begin{itemize}[noitemsep]
\item Trip to Sendai to attend Ruby Kaigi
\item Researching about Ruby through books
\item Coming up with a project to work on
\item Writing an informal requirement document on features I wanted to add
\item Implementing and testing the project as I go
\item Final presentation towards every interns and professors that was involved in this program to demonstrate what I did
\end{itemize}

\subsection{What I've learned}

I've learned many things over this internship.  First, I learned about Ruby.  I learned about how the language works in the lower level, as well as very niche tools Ruby have to assist the programmers.  I've also learned about how the research labs are operated.  I attended weekly meetings to explain what I did that week and watched other students perform their own research.  Finally, I've learned about Japanese culture and how they work as a team.

I think this experience will be beneficial to me in the future, as my ultimate goal as an computer scientist is to develop a programming language, which is something I partially did in this internship program.  Additionally, because it was such an extensive project, I had to use various techniques I've learned as a student here, such as software development techniques, breadth first search algorithm, and finite state machine diagram, which helped me refresh my memory on how these concept work.  By applying concepts I learned from older courses, I think I've become a better student and better programmer overall.
