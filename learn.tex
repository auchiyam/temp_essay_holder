\section{Research}

Before I began working on my project, I spent the beginning few weeks learning about the programming language Ruby.  The learning process included going on a 3 day trip to Sendai to attend Ruby Kaigi and reading various books about the programming language Ruby.

\subsection{Ruby Kaigi}

The first thing I did when I started the intern was attending Ruby Kaigi.  Ruby Kaigi is an annual conference for the programming language Ruby that hosts various talks by different programmers ranging from talking about their on going project to educational lectures about niche features of Ruby.  In this conference, I have learned a lot about the language itself, such as its weaknesses and strengths and how people solve problems they encountered.  The most important thing I've learned about Ruby in this conference is most likely Binding and TracePoint.  Binding is an object representing the scope of the program.  It can retrieve information on things such as value of local variable and even running some codes within that scope using eval() method.  TracePoint is a set of events that trigger based on the code you are running.  For example, TracePoint can trigger every line of your code, when a method has been called, when the method returns, and many other methods.  It is intended to be used for debugging and logging, but like you can see in the future sections, it is a very powerful tool that can manipulate the code in the bizarre manners.  Without some of the knowledge I gained through Ruby Kaigi I may not have been able to implement many of the features in this program.

\subsection{Metaprogramming Ruby}

After the Ruby Kaigi has ended, the internship has officially begun.  The first thing I did when I got to the lab was to learn a bit more about the Ruby.  Thanks to the supervising professor's recommendation, I was able to pick up on the principles of metaprogramming quickly.  One book the professor recommended me was \textit{Metaprogramming Ruby} by Paolo Perrotta.  The book, like the title suggests, is about metaprogramming in Ruby.  It explained several different features about metaprogramming in Ruby in detail, such as aliasing, hook methods, eugenclasses, and how the classes and inheritance are structured.  This helped me learn more about the metaprogramming processes and other features that makes Ruby a powerful tool for DSL or domain specific languages.  Through this book, I picked up metaprogramming techniques and DSL principles, which I put to use in my project.  Although I have read other books like \textit{Ruby Under Microscope} by Pat Shaughnessy, Metaprogramming Ruby was the most influential to me in this project.

Now that I have spent some time learning about how Ruby operates, it was time to think about exactly what kind of project I wanted to work on.
